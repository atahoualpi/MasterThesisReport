\chapter{Subjective Experiments}\label{ch:tests}
When working in a field in which human perception plays a significant role, it is important to test whether the work produced makes sense for target users and if it solves successfully the problem that it was designed for.

To examine the quality of physics-based synthetic sounds on listeners, \textit{A-B} and \gls{MUSHRA} \cite{series2014method} tests were performed. The aim is to answer the following questions: \begin{inparaenum}[1)]
\item Which of the two synthesis methods (sinusoidal or filter-based additive synthesis) offers the best results? 
\item Does sound variation make an audio cue more appealing?
\item Which is the range (in \gls{Q} factor values) of every material's sound?
\end{inparaenum}
The results of the experiments are used to verify the modal synthesis methods implemented in this thesis, study the listener's reaction to sound variation and increase our understanding of spectral characteristics in relation with the material of the object. The interface of the tests is displayed in appendix \ref{ap:experiments}.

\section{Preparation}
The experiments use several audio files that were recorded inside the Unity\textsuperscript{\textregistered} platform. Two special scenes were created to test both the impulse response of the synthesis model and the sounds in real-life conditions - when an object is falling and colliding with other objects (figure \ref{fig:test_scenes}). 

In the first scene (figure \ref{fig:test_sc1}), the audio files for the first experiment were recorded. During the recording session, the cube was ``tagged'' with different object names (a process which implies assigning different modal data) and let it hit the ground without bouncing. The process was repeated adding an \textit{Audio Source} component to the cube with the anechoic chamber recordings to achieve consistency of the stimuli.

In the second scene (figure \ref{fig:test_sc2}), the audio files for the second and third experiment were recorded. In this scene, objects fall, roll and scratch freely on rotated platforms, simulating a real room with obstacles.

Every sound in the audio files starts one second after the participant presses play and ends half a second after no sound can be heard. They are recorded with 16-bit resolution and a sample rate of 44100Hz using Audacity\textsuperscript{\textregistered} \cite{bib:audacity}. Stimuli was presented to the participants through a pair of \textit{AKG K271} headphones, in a room with reduced external noise.

\begin{figure}[H]
    \centering
    \begin{subfigure}[b]{0.48\textwidth}
        \includegraphics[width=\textwidth]{impulsescene_r.PNG}
        \caption{Impulse Response Simulator Scene.}
        \label{fig:test_sc1}
    \end{subfigure}
    ~ %add desired spacing between images, e. g. ~, \quad, \qquad, \hfill etc. 
      %(or a blank line to force the subfigure onto a new line)
    \begin{subfigure}[b]{0.48\textwidth}
        \includegraphics[width=\textwidth]{usertestscene_r.PNG}
        \caption{Real-Life Conditions Scene.}
        \label{fig:test_sc2}
    \end{subfigure}
    \caption{The Unity\textsuperscript{\textregistered} scenes designed to record the audio files used for the user tests.}\label{fig:test_scenes}
\end{figure}

Table \ref{tab:test_dec} gives details about all parts of the experiment.

\begin{table}[H]
	\centering
    \begin{tabular}{ c  l  p{7cm}  }
    \toprule
    \textbf{Experiment} & \textbf{Scene} & \textbf{Description} \\ \toprule
    \addlinespace
    $1$ & Single impact (\ref{fig:test_sc1}) & Participants listen to impact sounds and choose which method (filter-based or sinusoidal additive synthesis) sounds closer to the real-world recording.  \\
    \addlinespace
    $2$ & Rotated platforms (\ref{fig:test_sc2}) & Participants listen to sounds recorded using three different methods: real-world recording, filter-based and sinusoidal additive synthesis. They are asked to decide, for each of them, whether a sound variation per object area or a single sound per object sounds better in their opinion.  \\
    \addlinespace
    $3$ & Rotated platforms (\ref{fig:test_sc2}) & Participants listen to sounds produced by five different objects (one of each material under observation: plastic, wood, ceramic, glass and metal). However, they listen to a range of sounds that derive from a continuous increment of a parameter that affects the material of the object. They are asked to choose the point where a change of the material took place. \\ 
    \bottomrule
    \end{tabular}
    \caption{Overview of the experiments.}
    \label{tab:test_dec}
\end{table} 

\section{Stimuli}
Three different tests were performed. In the first test, the participant listens to 44 impulse responses, corresponding to different areas of the eleven materials. Each trial of the test includes a reference sound, which is the recording of the actual sound produced by the physical object and the two different synthesized sounds, corresponding to the two examined methods. The participant is then asked to choose which of the two synthesized sounds is closer to the reference and to what extend. The goal of this experiment is to gather information about the quality of the two methods and address which one of the two is best.

In the second test, the stimuli consists of 33 trials that contain sounds produced by falling objects. This time each trial includes two sounds and no reference. One of the sounds corresponds to an object that has been split into ``sound areas'' and struck in different locations. The second sound corresponds to an object that does not present sound variations along its surface and that is also struck several times. This single sound was chosen instead of other recordings corresponding to the object because it was the closest to the real sound (authors' opinion). Participants were asked which of the two sounds they preferred and to what degree. This test provides the results for sound variation versus repeating the same sound over and over.

The stimuli of the third test are sounds coming from five different objects, one of each material tested in this thesis (plastic jug, wooden mortar, ceramic plate, glass bottle and metallic cooking pot). For each object, ten different sounds per synthesis method are used. Each trial includes ten sounds that correspond to the same event, but with a small variation of the \gls{Q} for every sound. More specifically, starting from a value of $1000$, the \gls{Q} was increased by $200$ up to $4600$. Some sounds that were too similar with others were removed to decrease the total size and length of the test. The participants were asked to choose the sounds where, in their opinion, a change in the material happened. This test was done to validate the chosen values for the \gls{Q}.

In the first two experiments, both the sequence of trials and the conditions (which method is A and which is B) are randomized. However, in experiment number three, an increase of the \gls{Q} factor was desired and thus the order of the sounds matters. Hence, only the sequence of the trials were randomized.

\section{Participants}
Eleven adult volunteers participated in this test which lasts about half an hour. They were five women and six men, aged $25$ to $61$ who stated having normal hearing. They have different backgrounds, two of them were musicians and the rest were not. An attempt to extract different results for the two groups would not give qualitative outcomes due to the reduced number of individuals in one of them.


\section{Test Results}\label{sec:testresults}
The participants' responses were obtained in ``.txt'' format and processed using Python programming language and the following results were acquired.


\subsection{First Experiment}
Results of the first experiment are shown in figure \ref{fig:test1}. The results were divided into smaller sub-graphs, one for each of the eleven objects, to examine the participants' preferences with respect to the synthesis methods per object. More precisely, their opinion on which of the two methods is closer to the recording of the real-world object (used as reference).

Participants were asked to move a slider between the integer values $-3$ and $3$. Positive values correspond to a preference for the filter-based method, negative ones to the sinusoidal method and zero values mean that both methods sound the same. Possible answers are displayed in table \ref{tab:test1_ans}. Every different tick of the x-axis of figure \ref{fig:test1} corresponds to a different ``sound area'' of the object. From left to right the areas go from the top to the bottom.

\begin{figure}[H]
  \centering
    \includegraphics[width=\textwidth]{test1.png}
      \caption{The mean values and standard deviations per location for all objects. Positive values give a preference to the filter-based method, while negative ones give preference to the sinusoidal method. Participants choises were 0: both sound the same, 1/-1: the chosen is slightly better, 2/-2: the chosen is better and 3/-3: the chosen is much better.}\label{fig:test1}
\end{figure}

\begin{table}[ht]
	\centering
    \begin{tabular}{ c  c  l  }
    \toprule
    \textbf{Slider value} & \textbf{Method} & \textbf{Amount of Preference} \\ \toprule
    \addlinespace
    $3$ & Filter-based & Much better  \\
    $2$ & Filter-based & Better \\
    $1$ & Filter-based & Slightly better \\ 
    \addlinespace
    $0$ & \multicolumn{2}{c}{Both sound the same} \\
    \addlinespace
    $-1$ & Sinusoidal & Slightly better \\ 
    $-2$ & Sinusoidal & Better \\ 
    $-3$ & Sinusoidal & Much better \\
    \addlinespace
    \bottomrule
    \end{tabular}
    \caption{Possible answers for the 1st experiment.}
    \label{tab:test1_ans}
\end{table}  

This experiment examines which of the two synthesis methods - filter-based modal synthesis or sinusoidal additive synthesis - used in this thesis is more accurate. From figure \ref{fig:test1} it can be seen that there is no universal answer for all objects. More specifically, there is a tendency for the sinusoidal method for the glass (bottle and wine glass) and the wooden (cutting board, mortar and rolling pin) objects. On the other hand, metal objects (cooking pot and wok) and plastic ones (jug and bowl) seem to sound more realistic with the filter-based method. Finally, for the ceramic objects (cup and plate) both methods have similar results. It is important to notice that ceramic material lays, also, in the middle of the \gls{Q} values. Those results are summed up in table \ref{tab:method_mat}.

\begin{table}[H]
	\centering
    \begin{tabular}{ c  c  c c c }
    \toprule
    \textbf{Plastic} & \textbf{Wood} & \textbf{Ceramic} & \textbf{Glass} & \textbf{Metal} \\ \toprule
    Filter-based & Sinusoidal & Both & Sinusoidal & Filter-based  \\
    \bottomrule
    \end{tabular}
    \caption{Preferred synthesis method per material.}
    \label{tab:method_mat}
\end{table} 

\subsection{Second Experiment}
For the second experiment the results were divided into sub-graphs for every one of the objects. They are presented in figure \ref{fig:test2}. This experiment examines the participants' preferences between an impact sound that varies depending on the struck location and a single impact sound per object.

\begin{table}[H]
	\centering
    \begin{tabular}{  c  c  l  }
    \toprule
    \textbf{Slider value} & \textbf{Sound Variation} & \textbf{Amount of Preference} \\ \toprule
    \addlinespace
    $3$ & Yes & Much better \\ 
    $2$ & Yes & Better \\ 
    $1$ & Yes & Slightly better \\ 
    \addlinespace
    $0$ & \multicolumn{2}{c}{Both sound the same} \\ 
    \addlinespace
    $-1$ & No & Slightly better \\ 
    $-2$ & No & Better \\ 
    $-3$ & No & Much better \\
    \addlinespace
    \bottomrule
    \end{tabular}
    \caption{Possible answers for the 2nd experiment.}
    \label{tab:test2_ans}
\end{table}

Participants were asked to use a slider similar to the previous test, with possible answers shown in the table \ref{tab:test2_ans}. Positive values correspond to a preference for sound variation, while negative ones correspond to a preference for having a single sound per object. The three different ticks on the x-axis correspond to the actual recording, the filter-based and the sinusoidal method respectively. %

\begin{figure}[H]
  \centering
    \includegraphics[width=\textwidth]{test2.png}
      \caption{The mean values and standard deviations per method for all objects. Positive values show a preference for sound variation, while negative ones show a preference for one single sound per object. Participants' choises were 0: both sound the same, 1/-1: the chosen is slightly better, 2/-2: the chosen is better and 3/-3: the chosen is much better.}\label{fig:test2}
\end{figure}

The aim of this experiment is to prove whether sound variation for objects within a game are desirable and whether it makes a difference. By choosing the positive values, participants would prove that having spatial variation for impact sounds is more appealing than just having one repeated sound. Although the results do not provide a clear conclusion, the participants' answers lean towards this belief. In other words, sounds from seven out of eleven objects were chosen in the positive range. In addition, the rest of the objects' sounds have an average of about zero, which means that both examined sounds are very similar. This might have been caused due to the lack of rotation during the fall of the object on the platforms, especially for the cooking pot and the plate. That is to say that there was probably not enough variation on the recorded sounds or the sequence of sounds was not the most appropriate, which might have influenced some of the participants answers.

\subsection{Third Experiment}

Figure \ref{fig:test3} shows the results of the third experiment. This one was held to validate the chosen values of the \gls{Q} factor that corresponds to each material. For the purpose of this thesis, one value per material that sounded closer to reality was chosen. Those selected values are shown in table \ref{tab:default_Q}.  

Participants heard a range of materials assigned to the same object every time and were asked to move a slider if a change was noticed. The four changes that they had to identify are: plastic to wood, wood to ceramic, ceramic to glass and glass to metal.

\begin{figure}[H]
  \centering
    \includegraphics[width=\textwidth]{test3.png}
      \caption{The chosen values of the quality factor that correspond to material change of the same object.}\label{fig:test3}
\end{figure}

By studying figure \ref{fig:test3}, it can be seen that the values chosen by the participants in the experiment are similar to the chosen ones by the authors. The first change (plastic to wood) is chosen to be at value $1600$, while our choice is $1500$. Then, there is a change around value $2200$ which does not correspond to any of the examined materials. The second change (wood to ceramic) was chosen by the majority of the participants to be at value $3000$ (the same as our choice). Afterwards, the change from ceramic to glass is split between the values $3400$ and $3600$, when $3500$ is our chosen value. Finally, the last change (glass to metal) lays on value $4000$, which corresponds to our choice. As far as the extra change noticed by the participants is concerned, opens a discussion on whether a sixth material to be placed between wood and ceramic in the \gls{Q} range should be included.
