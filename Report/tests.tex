\chapter{Subjective Experiments}\label{ch:tests}
When working on a field where human perception plays a significant role, it is important to test whether your work makes sense for target users and if it solves successfully the problem that was designed for.

To examine the immersion of real-time produced and physics-based sounds on game players, we performed \textit{MUSHRA}\cite{series2014method} tests to people. Our aim was to answer the following questions: \begin{inparaenum}[1)]
\item Which of the two synthesis methods (sinusoidal and filter-based additive synthesis) is closer to reality? 
\item Does physics-based synthesis make a sound more realistic and less boring?
\item Which is the range (in Q-factor values) of every material's sound?
\end{inparaenum}.
The results of the experiments are used to verify the modal synthesis methods implemented in this thesis, choose the more qualitative one and increase our understanding of spectral characteristics in relation with the material of the object. 

\section{Preparation}
For the experiments we used several audio files that we recorded inside the Unity\textsuperscript{\textregistered} platform. We designed two special scenes for this purpose, because we wanted to test both the impulse response of the synthesis model and the sounds in real-life conditions -when an object is falling and colliding with other objects (figure \ref{fig:test_scenes}). 

In the first scene (figure \ref{fig:test_sc1}), we recorded the audio files for the first experiment. During the recording session, we ``tagged'' the cube as different objects (a process that assigns to the cube different modal data) and let it touch the ground without any bounce. We also repeated the process with an Audio Source and the recordings files, to achieve consistency of the stimuli.

In the second scene (figure \ref{fig:test_sc2}), we recorded the audio files for the second and third experiment. In this scene, objects fall, roll and scratch freely on rotated platforms, simulating a real room with obstacles.

\begin{figure}[H]
    \centering
    \begin{subfigure}[b]{0.48\textwidth}
        \includegraphics[width=\textwidth]{impulsescene_r.PNG}
        \caption{Impulse Response Simulator Scene.}
        \label{fig:test_sc1}
    \end{subfigure}
    ~ %add desired spacing between images, e. g. ~, \quad, \qquad, \hfill etc. 
      %(or a blank line to force the subfigure onto a new line)
    \begin{subfigure}[b]{0.48\textwidth}
        \includegraphics[width=\textwidth]{usertestscene_r.PNG}
        \caption{Real-Life Conditions Scene.}
        \label{fig:test_sc2}
    \end{subfigure}
    \caption{The Unity\textsuperscript{\textregistered} scenes designed to record the audio files used for the user tests.}\label{fig:test_scenes}
\end{figure}

Every sound in the audio files starts 1 second after participant presses play and ends half a second after no sound can be heard. They are recorded with 16-bit resolution and a sample rate of 44100Hz using Audacity\textsuperscript{\textregistered} \cite{bib:audacity}. Stimuli was presented to the participants through a pair of \textit{AKG K271} headphones, in a room with reduced external noise.

Table \ref{tab:test_dec} gives details about all parts of the experiment.

\begin{table}[H]
	\centering
    \begin{tabular}{ c  l  p{7cm}  }
    \toprule
    \textbf{Experiment} & \textbf{Scene} & \textbf{Description} \\ \toprule
    \addlinespace
    $1$ & Single impact (\ref{fig:test_sc1}) & Participants listen to impact sounds and choose which method (filter-based or sinusoidal additive synthesis) sounds closer to the real-world recording.  \\
    \addlinespace
    $2$ & Slanting platforms (\ref{fig:test_sc2}) & Participants listen to sounds recorded using 3 different methods: real-world recording, filter-based and sinusoidal additive synthesis. They are asked to decide, for each of them, whether a sound variation per object area or a single sound per object sounds better in their opinion.  \\
    \addlinespace
    $3$ & Slanting platforms (\ref{fig:test_sc2}) & Participants listen to sounds produced by 5 different objects (one of each material under observation: plastic, wood, ceramic, glass and metal). However, they listen to a range of sounds that derive from a continuous increment of a parameter that affects the material of the object. They are asked to choose the point where a change of the material took place. \\ 
    \bottomrule
    \end{tabular}
    \caption{Overview of the experiments.}
    \label{tab:test_dec}
\end{table} 

\section{Stimuli}
We performed three different tests. In the first test, the participant listens to 44 impulse responses, corresponding to different areas of the eleven materials. Each trial of the test includes a reference sound, which is the recording of the actual sound produced by the physical object and the two different synthesized sounds, corresponding to the two examined methods. He is then asked to choose which of the two synthesized sounds is closer to the reference and whether it is very or less close. The goal of this experiment was to gather information about the quality of the two methods and addresses the best of the two.

In the second test, the stimuli consists of 33 trials that contain sounds produced by falling objects. Each trial includes one sound that is produced when object is split into ``sound areas'' and each area produces a different sound and a second sound that is produced when every point of the object makes the same sound. This single sound was chosen to be the one produced from the area of each object where the recording taken (section \ref{sec:recordings}) was the closest to the real sound in our opinion. Participants were asked which of the two they preferred and how much in a scale from ``A is the same as B" to ``A is much better than B". This test provided us with the immersion results of sound variation within one object.

The stimuli of the third test are sounds coming from five different objects, one of each material under testing (plastic jug, wooden mortar, ceramic plate, glass bottle and metallic cooking pot). For each object, we use 10 different sounds per synthesis method. Each trial includes 10 sounds that correspond to the same event, but with a small variation on the Q-factor for every sound. More specifically, starting from a value of 1000, we increased the Q-factor by 200 up to 4600, removed some sounds that were too similar with others to decrease the size of the test and provided participants with the rest. They were asked to choose the sounds where, in their opinion, a change in the material happened. We performed this test to validate the chosen values for the Q-factor.

In the two first experiments, both the sequence of trials and the conditions (which method is A and which is B) are randomized. However, in experiment number three, we wanted to keep an increasing habit on the Q-factor, otherwise it would not make sense. Hence we randomized only the sequence of the trials.

\section{Participants}
The participants are nine adult volunteers. Three women and six men, aged $25$ to $61$. They stated normal hearing. They have different backgrounds, namely a mixture of audio specific and non-audio specific participants. An attempt to extract different results for the two groups would not give qualitative outcomes, since only two of the participants are engaged with audio (a musician and a \gls{DSP} Engineer).

\section{Test Results}
The participants' responses were processed using python programming language and the following results were acquired.

\subsection{1st Experiment}
Results of the 1st experiment are shown in figure \ref{fig:test1}. We divided the results into smaller sub-graphs, one for each of the eleven objects, to examine the participants' preference in the synthesis methods per object. Particularly, their opinion on which of the two methods resembles more the original recording of the real-world object (used as reference).

\begin{figure}[H]
  \centering
    \includegraphics[width=\textwidth]{test1.png}
      \caption{The mean values and standard deviations per location for all objects. Positive values give a preference to the filter-based method, while negative ones give preference to the sinusoidal method. Participants choises were 0: both sound the same, 1/-1: the chosen is slightly better, 2/-2: the chosen is better and 3/-3: the chosen is much better.}\label{fig:test1}
\end{figure}

Participants were asked to move a slider between the integer values $-3$ and $3$. Positive values correspond to a preference in the filter-based method, negative to the sinusoidal method and zero values means that both methods sound the same. Possible answers are displayed in table \ref{tab:test1_ans}. Every different tick of the x-axis corresponds to a different ``sound area'' of the object. From left to right the areas go from the top to the bottom.

\begin{table}[H]
	\centering
    \begin{tabular}{ c  c  l  }
    \toprule
    \textbf{Slider value} & \textbf{Method} & \textbf{Amount of Preference} \\ \toprule
    \addlinespace
    $3$ & Filter-based & Much better  \\
    $2$ & Filter-based & Better \\
    $1$ & Filter-based & Slightly better \\ 
    \addlinespace
    $0$ & \multicolumn{2}{c}{Both sound the same} \\
    \addlinespace
    $-1$ & Sinusoidal & Slightly better \\ 
    $-2$ & Sinusoidal & Better \\ 
    $-3$ & Sinusoidal & Much better \\
    \addlinespace
    \bottomrule
    \end{tabular}
    \caption{Possible answers for the 1st experiment.}
    \label{tab:test1_ans}
\end{table}  

With this test we wanted to examine which of the two synthesis methods - filter based modal synthesis or sinusoidal additive synthesis - used in this thesis is more accurate. From figure \ref{fig:test1} we can see that one universal answer for all objects does not exist. More specifically, we can see a tendency for the sinusoidal method for the glass (bottle and wine glass) and the wooden (cutting board, mortar and rolling pin) objects. On the other hand, metal objects (cooking pot and wok) and plastic ones (jug and bowl), seem to sound more realistic in the filter-based method. Finally, for the ceramic objects (cup and plate) both methods have similar results. It is important to notice that ceramic material lays, also, in the middle of the Q-factor values. Those results are summed up in table \ref{tab:method_mat}.

\begin{table}[H]
	\centering
    \begin{tabular}{ c  c  c c c }
    \toprule
    \textbf{Plastic} & \textbf{Wood} & \textbf{Ceramic} & \textbf{Glass} & \textbf{Metal} \\ \toprule
    Filter-based & Sinusoidal & Both & Sinusoidal & Filter-based  \\
    \bottomrule
    \end{tabular}
    \caption{Preferred synthesis method per material.}
    \label{tab:method_mat}
\end{table} 

\subsection{2nd experiment}
For the second experiment we also divided the results into sub-graphs per object. They are presented in figure \ref{fig:test2}. This experiment examines the participants' preference between a sound variation on impact sounds depending on the hit point and a single impact sound per object.

\begin{table}[H]
	\centering
    \begin{tabular}{  c  c  l  }
    \toprule
    \textbf{Slider value} & \textbf{Sound Variation} & \textbf{Amount of Preference} \\ \toprule
    \addlinespace
    $3$ & Yes & Much better \\ 
    $2$ & Yes & Better \\ 
    $1$ & Yes & Slightly better \\ 
    \addlinespace
    $0$ & \multicolumn{2}{c}{Both sound the same} \\ 
    \addlinespace
    $-1$ & No & Slightly better \\ 
    $-2$ & No & Better \\ 
    $-3$ & No & Much better \\
    \addlinespace
    \bottomrule
    \end{tabular}
    \caption{Possible answers for the 2nd experiment.}
    \label{tab:test2_ans}
\end{table}

Participants were asked to use a slider similar to the previous test, with possible answers shown in the table \ref{tab:test2_ans}. Positive values correspond to preference in sound variation, while negative ones correspond to preference in single sound per object. The three different ticks on the x-axis correspond to the actual recording, the filter-based and the sinusoidal method respectively. %

\begin{figure}[H]
  \centering
    \includegraphics[width=\textwidth]{test2.png}
      \caption{The mean values and standard deviations per method for all objects. Positive values give a preference to sound variation, while negative ones give preference to one single sound per object. Participants choises were 0: both sound the same, 1/-1: the chosen is slightly better, 2/-2: the chosen is better and 3/-3: the chosen is much better.}\label{fig:test2}
\end{figure}

The aim of this experiment is to prove whether sound variation in game sound effects is desirable and makes a difference. By choosing the positive values, participants would prove that having spatial variation in impact sounds benefits the immersion of the player. Although, it is not very profound, participants' answers lean towards this belief. In other words, sounds from seven out of eleven objects were chosen in the positive range. In addition, the rest of the objects' sounds have an average of about zero, which means that both examined sounds are the same. This might have been caused due to the lack of rotation during the fall of the object on the platforms, especially for the cooking pot and the plate. That is to say, probably there was no actual variation on the recorded sounds.

\subsection{3rd experiment}
\Todo{check if it comes out that we cannot transform a given object's material just with damping (maybe we need to change the spectral content)}

Figure \ref{fig:test3} shows the results of the third experiment. This one was held to validate the chosen values of the Q-factor that correspond to each material. For the purpose of this thesis, we chose one value per material that sounded closer to reality, in our opinion. Those selected values are shown in table \ref{tab:default_Q}.  

Participants heard a range of materials assigned to the same object every time and were asked to move a slider if a change was noticed. The four changes that they had to identify are: plastic to wood, wood to ceramic, ceramic to glass and glass to metal.

\begin{figure}[H]
  \centering
    \includegraphics[width=\textwidth]{test3.png}
      \caption{The chosen values of the quality factor that correspond to material change of the same object.}\label{fig:test3}
\end{figure}

By studying figure \ref{fig:test3}, we can see that the values chosen from the experiment participants validate ours. The first change (plastic to wood) is chosen to be at value 1600, while our choice is 1500. Then, there is a change around value 2200 which does not correspond to any of our examined materials. The second change (wood to ceramic) was chosen by the majority of the participants to be at value 3000 (the same as our choice). Afterwards, the change from ceramic to glass is split between the values 3400 and 3600, when 3500 is our chosen value. Finally, the last change (glass to metal) lays on value 4000, which corresponds to our choice. As far as the extra change noticed by the participants is concerned, opens a discussion of whether we should include a sixth material to be placed between wood and ceramic in the Q-factor range.

