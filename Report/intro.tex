\chapter{Introduction}

\paragraph{Immersion and all these stuff that makes our thing good. Why we are doing it and what do we want to give to the community?\\}

Audio in interactive projects like video games and VR/AR applications, plays a significant role for user immersion and realism. Visual and acoustic experiences are interconnected and lacking one of them spoils the whole experience. 

The most difficult task is to produce realistic virtual sounds inside the application, difficult to distinguish them from the real ones. This can be achieved not only by playing back a realistic sound, but also by taking care of the environment effects and the context. For example, striking a nail on a board when it still vibrates from the previous struct, produces a different sound that gets added to the previous one \cite{Cook:2002:RSS:515316}.

Although some sounds like the soundtrack music or voices can be recorded and played back, sound effects need physically-based methods to synthesize them real-time, so as to be realistic and accurate. All objects vibrate when struck, even solid ones. It is something non-noticeable from a human eye, but it is capable of generating sound.  

\colorbox{pink}{Stuff about 3d sound as well.}

\paragraph{Stuff about sound in general and description of the thesis\\}

Sounds are strongly related to our everyday life and the ways we understand things. Through our experience, we can visualize an event by only hearing the sound it produces (e.g. a car approaching). The vibration caused by the collision of two objects produces sound that depends on the collision force, the duration of the interaction  and the changes over time of it, but also on the size, shape, material and texture of the two objects. All these attributes form a unique sound and the sound waves produced from the interaction give the information to the listener \cite{gaver1993world}.

In this thesis we present an audio design tool made for Unity\textregistered software platform, for physics-based sound synthesis in virtual environments. \\
\Todo{describe how much better it is to have procedural audio than pre-recorded sounds and on top of that physics-based sounds!! WOW!!}

By pre-computing all necessary data, we are able to model a sound produced by a 3D model, very similar to the one that would be produced from a real-world one.

\paragraph{Why is our method better that others? (eg wavetable)? And why we think this is the future of the audio in video games?\\}
  



