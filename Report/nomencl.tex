%==================================================================================================
% Abbreviations and Nomenclature
%==================================================================================================
% abbreviations:
\newacronym{AR}{AR}{Augmented Reality}
\newacronym{BPF}{BPF}{Band-Pass Filter}
\newacronym{BW}{BW}{Bandwidth}
\newacronym{DAC}{DAC}{Digital-to-Analog Converter}
\newacronym{DLL}{DLL}{Dynamic-Link Library}
\newacronym{DTU}{DTU}{Denmark's Technical University}
\newacronym{FBX}{FBX}{Filmbox}
\newacronym{FEM}{FEM}{Finite Element Method}
\newacronym{FFT}{FFT}{Fast Fourier Transform}
\newacronym{GUI}{GUI}{Graphical User Interface}
\newacronym{MUSHRA}{MUSHRA}{MUltiple Stimuli with Hidden Reference and Anchor}
\newacronym{Pd}{Pd}{Pure Data}
\newacronym{UI}{UI}{User Interface}
\newacronym{VR}{VR}{Virtual Reality}
\newacronym{CPU}{CPU}{Central Processing Unit}
\newacronym{AI}{AI}{Artificial Intelligence}
\newacronym{FPS}{FPS}{Frames Per Second}
\newacronym{ADSR}{ADSR}{Attack, Decay, Sustain, Release (sound envelope)}
\newacronym{DSP}{DSP}{Digital Signal Processing}
\newacronym{OSC}{OSC}{Open Sound Control}
\newacronym{VBAP}{VBAP}{Vector Base Amplitude Panning}
\newacronym{HRTF}{HRTF}{Head-Related Transfer Function}
\newacronym{Q}{Q}{Q factor}
\newacronym{WFS}{WFS}{Wave Field Synthesis}
% nomenclature:
\newglossaryentry{freq}{
  name = $f$ ,
  description = Frequency,
}

\newglossaryentry{damping}{
  name = $d$ ,
  description = Decay rate,
}

\newglossaryentry{amplitude}{
  name = $A$ ,
  description = Matrix of oscillating amplitudes on every point of an object,
}

\newglossaryentry{modalMatrix}{
  name = $M$ ,
  description = Matrix of modal data,
}

\newglossaryentry{Qfactor}{
  name = $Q$ ,
  description = Parameter describing the damping of an oscillator,
}

\newglossaryentry{Dparameter}{
  name = $D$ ,
  description = Parameter corresponding to type of material,
}

\newglossaryentry{KineticEnergy}{
  name = $K$ ,
  description = Kinetic energy,
}

\newglossaryentry{ForceMagnitude}{
  name = $CFM$ ,
  description = Collision force magnitude,
}

\newglossaryentry{angularVel}{
  name = $\omega$ ,
  description = Angular velocity,
}



