\chapter{Conclusion}

A sound designing tool was presented in this study, where two different synthesis algorithms were implemented. The goal was to achieve physics-based procedural audio implementation into games using the Unity\textsuperscript{\textregistered} game engine. 

A number of everyday objects were used, which were separated into different ``sound areas'' and an impact sound for each one of them was recorded. In addition, 3D models of the objects were created, matching their shape and dimensions. By using a peak detection algorithm on the real-world recordings, we were able to extract modal data necessary for sound re-production. Those data were fed to the synthesis algorithms, together with an impulse that activates the sound synthesis procedure. This process made it successful to transfer the audio properties of the original objects to the corresponding virtual ones.

Both synthesis methods include three different parts; an impact, a rolling and a scratching sound synthesis, which summarize all possible interactions of two surfaces. The tool's \gls{UI}, with high level adjustable parameters, enables the designers to customize the sounds according to their needs.

Although the extraction of data from recordings offers computational efficiency, it is impossible to include the whole sound information without alterations. Thus, the synthesized sounds do not sound exactly like the reference ones, but their source is recognizable. 