\chapter{Theoretical Background}\label{ch:theory}
% short overview of the theory parts
\mbox{}\par
This is a way to link to explanations \gls{DOA} 

THis is a todo: 
\Todo{do smth}\\
THis is smth done:
\done\Todo{this is done}

\section{State-Of-The-Art}\label{sec:state_art}
\mbox{}\par

\section{Modal Analysis}\label{sec:modal_analysis}
\mbox{}\par
In this thesis we are using solid objects that are struck in different ways to produce sound. These ways could be falling on the floor or colliding with another object. The sounds produced can be impact, rolling or scratching sounds. When an object is struck, the forces applied cause deformations to it, emitting sound waves through the vibration of its outer surfaces \cite{van2001foleyautomatic}.

Modal analysis studies the response of models under excitation. It uses the 3D model of an object to calculate its modal modes (vibration modes). There are multiple ways to do this, with the most accurate being FEM (Finite Element Method). The objective of FEM is to calculate the natural frequencies of a structure when it vibrates freely.

\subsection{Features Extraction}\label{sec:features_extract}
\mbox{}\par
Modal analysis is performed before modal synthesis, to extract the necessary data. Modal synthesis is the sum of damped oscillators each corresponding to a modal frequency, as it will be discussed further below. The data needed for synthesis are shown in the table below:

\begin{center}
    \begin{tabular}{ | l | l | l | p{5cm} |}
    \hline
    \textbf{Symbol} & \textbf{Description} & \textbf{Derivation} \\ \hline
    $A_n$ & Initial amplitude & Force proximity \\ \hline
    $\delta_n$ & Damping & Material properties \\ \hline
    $\omega_n$ & Partial frequency & Modal analysis \\
    \hline
    \end{tabular}
\end{center} 
 
\section{Modal Synthesis}\label{sec:modal_synth}
\mbox{}\par 


\subsection{Sinusoidal Additive Synthesis}\label{sec:sin_synth}
\mbox{}\par
\subsection{Filter-based Modal Synthesis}\label{sec:add_synth}
\mbox{}\par

\begin{equation}\label{eq:spherical_wave}
 \frac{1}{r^2}\frac{\partial}{\partial r}(r^2 \frac{\partial p}{\partial r})+\frac{1}{r^2 \text{sin}\theta}\frac{\partial}{\partial \theta}(\text{sin}\theta\frac{\partial p}{\partial \theta}) + \frac{1}{r^2 \text{sin}^2\theta}\frac{\partial^2 p}{\partial \phi^2}-\frac{1}{c^2}\frac{\partial^2 p}{\partial t^2} = 0.
\end{equation}
 \cite{bib:fourierAcoustics}.