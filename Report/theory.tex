\chapter{Theoretical Background}\label{ch:theory}
Short overview of the theory parts\\
This is a way to link to explanations \gls{DOA} 

THis is a todo: 
\Todo{todo test}\\
THis is smth done:
\done\Todo{this is done}

\section{State-Of-The-Art}\label{sec:state_art}
\textit{Wildes and Richards (1988)} defined the angle of internal friction ($\phi$), a shape-invariant acoustical parameter that heuristically categorizes sounds into material categories, as
\begin{equation}
\tan(\phi) = \alpha / \pi f,
\end{equation}
where $\alpha = \tau_e$ is the damping coefficient, with $\tau_e$ being the time for the vibration amplitude to decay to $1/e$ of its original value after the object is struck  and $f$ is the vibration frequency\cite{giordano2006material}.

\section{Modal Analysis}\label{sec:modal_analysis}
In this thesis we are using solid objects that are struck in different ways to produce sound. These ways could be falling on the floor or colliding with another object. The sounds produced can be impact, rolling or scratching sounds. When an object is struck, the forces applied cause deformations to it, emitting sound waves through the vibration of its outer surfaces \cite{van2001foleyautomatic}.

Modal analysis studies the response of models under excitation. It uses the 3D model of an object to calculate its modal modes (vibration modes). There are multiple ways to do this, with the most accurate being FEM (Finite Element Method). The objective of FEM is to calculate the natural frequencies of a structure when it vibrates freely.

Another method for modal analysis is the ``Exampled-guided'', where data get extracted using example recordings of the objects being struck. Using a suitable algorithm it is easy to extract features from the recordings such as the fundamental frequency and its harmonics and the frequency peaks of the signal.

\subsection{Data Extraction}\label{sec:data_extract}
Modal analysis is performed before modal synthesis, to extract the necessary data. Modal synthesis is the sum of damped oscillators each corresponding to a modal frequency, as it will be discussed further below. The data needed for synthesis are shown in the table \ref{tab:extracted_data}.

\begin{table}
	\centering
    \begin{tabular}{ | l | l | l | p{5cm} |}
    \hline
    \textbf{Symbol} & \textbf{Description} & \textbf{Derivation} \\ \hline
    $A_n$ & Initial amplitude & Modal analysis \\ \hline
    $d_n$ & Damping & Material properties \\ \hline
    $f_n$ & Modal frequency & Modal analysis \\
    \hline
    \end{tabular}
    \caption{Data extracted in modal analysis.}
    \label{tab:extracted_data}
\end{table} 

Since every different point being struck produces different deformations on the object, we need matrices of size $N$ ($N$ being the number of struck points of the object). More specifically, we need a vector $\textbf{f}$ of size $\textbf{N}$ corresponding to the modal frequencies of every point, a vector $\textbf{d}$ of size $\textbf{N}$ corresponding to the \colorbox{pink}{damping ratios} and a matrix $\textbf{A}$ of size $\textbf{NxK}$, where $K$ is the number of modal frequencies calculated in one point, which corresponds to the amplitudes of each mode in every point of the object. All the above gives the modal model which can be symbolized as $\textbf{\textit{M = \{f, d, A\}}}$ \cite{van2001foleyautomatic}.
 
\section{Modal Synthesis}\label{sec:modal_synth}
In the modal synthesis part, using the data extracted above, we synthesize the struck sound corresponding to the object. There are different ways to synthesize impact sounds, two of them being ``Sinusoidal Additive Synthesis'' and ``Filter-based Modal Synthesis''. The former uses exponential damping and the latter band-pass filters where the damping is the Q-factor of the filter. 

\subsection{Sinusoidal Additive Synthesis}\label{sec:sin_synth}
At a struck point $k$ when vibrating in mode $n$, the impulse response of the model is:
\begin{equation}\label{eq:modal_response}
y_k = \sum\limits_{n=1}^{N} A_{nk}\ e^{-d_n t}\ \cos(2 \pi f_nt)
\end{equation}
if $t>0$ and $y_k = 0$ if $t<=0$ \cite{van2001foleyautomatic}.

\begin{figure}[H]
  \centering
    \includegraphics[width=0.5\textwidth]{sinusoidal_add_synth.PNG}
      \caption{Sinusoidal Additive Synthesis Algorithm \cite{Cook:2002:RSS:515316}.}
      \label{fig:sin_add_synth}
\end{figure}

\subsection{Filter-based Modal Synthesis}\label{sec:add_synth}

\paragraph{Band-pass Filters\\}\label{par:bpf}

At this point we will give some basic description of the band-pass filter since it is widely used in this thesis. Band-pass filters (BPFs) take a signal as input and give only a range of it as output, attenuating the rest of the frequencies. This range depends on the central frequency $f_c$. A filter of this kind is a result of a cascading of a low-pass and a high-pass filter circuit.

The passing range or ``band'' of frequencies is called \textbf{Bandwidth (BW)}. Defining as 0db the resonant peak, we can find the two cut-off frequencies ($f_{c{_\textsc{lower}}}$ and $f_{c_{\textsc{higher}}}$) at -3dB. The range between them is the bandwidth (equation \ref{eq:bw}). In figure \ref{fig:resp_bpf} we can see the frequency response of a BPF. \cite{bib:bpf}. 
\begin{equation}\label{eq:bw}
BW = f_{c_{\textsc{higher}}}-f_{c_{\textsc{lower}}}
\end{equation}   

\begin{figure}[H]
  \centering
    \includegraphics[width=0.7\textwidth]{BPF.PNG}
      \caption{Frequency Response of a Band-pass Filter  \cite{bib:bpf}.}
      \label{fig:resp_bpf}
\end{figure}

\paragraph{Synthesis\\}\label{par:synth}

This method is also additive, since we are adding the outputs of a number of band-pass filters. To synthesize a sound using this method, we use as many filters as the modal frequencies. The filter takes as input an impulse, the center frequency which is the modal frequency and a \textbf{Quality factor (Q-factor)} which specifies the bandwidth of the filter. The Q-factor is calculated heuristically, depending on the material of the sound and is inversely proportional to the bandwidth ($Q=f_c/BW$), so the lower the Q-factor, the wider the bandwidth and vice-versa. Hence, more and less frequencies respectively will be included in the \colorbox{pink}{audible} range. We call the above structure a \textit{resonator}, which also includes a multiplication with the corresponding amplitude, taken from the $A$ matrix.

\begin{figure}[H]
  \centering
    \includegraphics[width=0.5\textwidth]{filter-based_add_synth.PNG}
      \caption{Filter-based Modal Synthesis Algorithm \cite{Cook:2002:RSS:515316}.}
      \label{fig:filter_synth}
\end{figure}

